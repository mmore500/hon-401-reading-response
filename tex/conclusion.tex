As the American frontier receded behind a steady influx of settlers, religious practice had to adapt to changing circumstances.
Chaos characterized this process of religious transition -- the trailing edge of the frontier, where settlement began to reach critical mass necessary to precipitate civilized communities, was a hotbed of religious activity marked by an ``expansive tendency'' that led to a ``contest for power'' between sects  \cite{Turner1894TheHistory}.
McCarthy puts the group-cohesion horizontal functionality that religion begins to gain during this transition on display in Nacogdoches, the frontier town the kid stumbles upon at the opening of \textit{Blood Meridian}.
Destabilized by the judge's salacious accusations, the institution of religion quite literally crumbles and along with it crumbles the delicate social order in Nacodgdoches.
\blockcquote[p 7]{McCarthy1992BloodWest}{Already gunfire was general within the tent \ldots and people were pouring out, women screaming, folk stumbling, folk trampled underfoot in the mud \ldots the tent began to sway and buckle and like a huge and wounded medusa it slowly settled to the ground \ldots}. Through this striking imagery, McCarthy displays an intertwined relationship between religion and social order.
However fledgling, in the setting of Nagodoches religion religion can be observed beginning to influence the social dynamics of the town's residents.
Alongside this horizontal shift, the vertical elements of religion are simultaneously shifting to support the new horizontal role centered on behavioral regulation. This vertical shift is evinced by the cautionary sinner-shaming oratory employed by the tent preacher in Nagodoches. Hence, McCarthy depicts the horizontal and vertical components of religion in a state of flux as the focus of horizontal functionality moves away from individuals' ability to cope with hardship \cite[p 6]{McCarthy1992BloodWest}. Doubtlessly to the delight of evolutionists such as Wilson, \textit{Blood Meridian} demonstrates that religion, analogously to biological life, adapts its form and function, its vertical and horizontal dimensions, to changing circumstances over time.
