At and beyond the bleak and austere frontier, like the sunbleached bones of cattle that scatter the landscape of Comac McCarthy's \textit{Blood Meridian}, the true nature of humanity is laid bare.
In particular, the frontier sheds light on human religion.
At first blush in more comfortable and stable environments, religion seems superfluous to a scientific observer.
However, through a functionalist lens inspired by evolutionary biology thinking, the frontier demonstrates the practical utility of religion.
McCarthy's \textit{Blood Meridian} provides a collection of striking vignettes that can help elucidate the practical roles of religion at and beyond the frontier.
Through this functionalist lens, analysis reveals that at and beyond the frontier the functionality of religion shifts away from regulating group dynamics, a major role of religion in more densely populated areas, towards helping individuals endure excruciating and arbitrary misfortune.

The functional framework for analysis of religion we will employ was developed by evolutionary biologist David Sloan Wilson.
He posits that religion can be decomposed into horizontal and vertical components.
In this scheme, vertical components of a religion describe a believer's relation to a higher power while horizontal components refer to how believers relate with one another and the world around them \cite[p 255]{Wilson2007EvolutionLives}.
Wilson's key insight is that the vertical dimension of religion, which to a scientific observer often appears arbitrary and contrived, can affect control over the horizontal dimension.
For example, believing in an all-knowing, perfect higher power who commands you to act by a certain set of commandments can be an effective way of promoting resource sharing behavior.
In other words, Wilson suggests an analysis of metaphysical religious doctrine guided by the question \blockcquote[p 256]{Wilson2007EvolutionLives}{
what does it cause people to do?
}.
Wilson is not alone in considering this functionalist perspective on religion.
Benjamin Franklin, a member of the Deist movement in early America, recorded similar observations on religion in his autobiography:
\blockcquote[p 59]{Franklin2008AutobiographyWritings}{
I entertain'd an Opinion, that tho' certain actions might not be bad \textit{because} they were forbidded by it, or \textit{good} because it commanded them; yet probably those Actions might be forbidden \textit{because} they were beneficial to us, in their own Natures, all the Circumstances of of things considered
}.
Franklin, like Wilson, arrives at the conclusion that religious doctrine might exist simply to influence behavioral outcomes and, presciently, Franklin makes the crucial observation that what actions are ``good'' and ``bad'' for people inherently depends on their circumstances or, as Wilson might term it, their environment.
